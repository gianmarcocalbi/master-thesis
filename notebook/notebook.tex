
% Default to the notebook output style

    


% Inherit from the specified cell style.




    
\documentclass[11pt]{article}

    
    
    \usepackage[T1]{fontenc}
    % Nicer default font (+ math font) than Computer Modern for most use cases
    \usepackage{mathpazo}

    % Basic figure setup, for now with no caption control since it's done
    % automatically by Pandoc (which extracts ![](path) syntax from Markdown).
    \usepackage{graphicx}
    % We will generate all images so they have a width \maxwidth. This means
    % that they will get their normal width if they fit onto the page, but
    % are scaled down if they would overflow the margins.
    \makeatletter
    \def\maxwidth{\ifdim\Gin@nat@width>\linewidth\linewidth
    \else\Gin@nat@width\fi}
    \makeatother
    \let\Oldincludegraphics\includegraphics
    % Set max figure width to be 80% of text width, for now hardcoded.
    \renewcommand{\includegraphics}[1]{\Oldincludegraphics[width=.8\maxwidth]{#1}}
    % Ensure that by default, figures have no caption (until we provide a
    % proper Figure object with a Caption API and a way to capture that
    % in the conversion process - todo).
    \usepackage{caption}
    \DeclareCaptionLabelFormat{nolabel}{}
    \captionsetup{labelformat=nolabel}

    \usepackage{adjustbox} % Used to constrain images to a maximum size 
    \usepackage{xcolor} % Allow colors to be defined
    \usepackage{enumerate} % Needed for markdown enumerations to work
    \usepackage{geometry} % Used to adjust the document margins
    \usepackage{amsmath} % Equations
    \usepackage{amssymb} % Equations
    \usepackage{textcomp} % defines textquotesingle
    % Hack from http://tex.stackexchange.com/a/47451/13684:
    \AtBeginDocument{%
        \def\PYZsq{\textquotesingle}% Upright quotes in Pygmentized code
    }
    \usepackage{upquote} % Upright quotes for verbatim code
    \usepackage{eurosym} % defines \euro
    \usepackage[mathletters]{ucs} % Extended unicode (utf-8) support
    \usepackage[utf8x]{inputenc} % Allow utf-8 characters in the tex document
    \usepackage{fancyvrb} % verbatim replacement that allows latex
    \usepackage{grffile} % extends the file name processing of package graphics 
                         % to support a larger range 
    % The hyperref package gives us a pdf with properly built
    % internal navigation ('pdf bookmarks' for the table of contents,
    % internal cross-reference links, web links for URLs, etc.)
    \usepackage{hyperref}
    \usepackage{longtable} % longtable support required by pandoc >1.10
    \usepackage{booktabs}  % table support for pandoc > 1.12.2
    \usepackage[inline]{enumitem} % IRkernel/repr support (it uses the enumerate* environment)
    \usepackage[normalem]{ulem} % ulem is needed to support strikethroughs (\sout)
                                % normalem makes italics be italics, not underlines
    

    
    
    % Colors for the hyperref package
    \definecolor{urlcolor}{rgb}{0,.145,.698}
    \definecolor{linkcolor}{rgb}{.71,0.21,0.01}
    \definecolor{citecolor}{rgb}{.12,.54,.11}

    % ANSI colors
    \definecolor{ansi-black}{HTML}{3E424D}
    \definecolor{ansi-black-intense}{HTML}{282C36}
    \definecolor{ansi-red}{HTML}{E75C58}
    \definecolor{ansi-red-intense}{HTML}{B22B31}
    \definecolor{ansi-green}{HTML}{00A250}
    \definecolor{ansi-green-intense}{HTML}{007427}
    \definecolor{ansi-yellow}{HTML}{DDB62B}
    \definecolor{ansi-yellow-intense}{HTML}{B27D12}
    \definecolor{ansi-blue}{HTML}{208FFB}
    \definecolor{ansi-blue-intense}{HTML}{0065CA}
    \definecolor{ansi-magenta}{HTML}{D160C4}
    \definecolor{ansi-magenta-intense}{HTML}{A03196}
    \definecolor{ansi-cyan}{HTML}{60C6C8}
    \definecolor{ansi-cyan-intense}{HTML}{258F8F}
    \definecolor{ansi-white}{HTML}{C5C1B4}
    \definecolor{ansi-white-intense}{HTML}{A1A6B2}

    % commands and environments needed by pandoc snippets
    % extracted from the output of `pandoc -s`
    \providecommand{\tightlist}{%
      \setlength{\itemsep}{0pt}\setlength{\parskip}{0pt}}
    \DefineVerbatimEnvironment{Highlighting}{Verbatim}{commandchars=\\\{\}}
    % Add ',fontsize=\small' for more characters per line
    \newenvironment{Shaded}{}{}
    \newcommand{\KeywordTok}[1]{\textcolor[rgb]{0.00,0.44,0.13}{\textbf{{#1}}}}
    \newcommand{\DataTypeTok}[1]{\textcolor[rgb]{0.56,0.13,0.00}{{#1}}}
    \newcommand{\DecValTok}[1]{\textcolor[rgb]{0.25,0.63,0.44}{{#1}}}
    \newcommand{\BaseNTok}[1]{\textcolor[rgb]{0.25,0.63,0.44}{{#1}}}
    \newcommand{\FloatTok}[1]{\textcolor[rgb]{0.25,0.63,0.44}{{#1}}}
    \newcommand{\CharTok}[1]{\textcolor[rgb]{0.25,0.44,0.63}{{#1}}}
    \newcommand{\StringTok}[1]{\textcolor[rgb]{0.25,0.44,0.63}{{#1}}}
    \newcommand{\CommentTok}[1]{\textcolor[rgb]{0.38,0.63,0.69}{\textit{{#1}}}}
    \newcommand{\OtherTok}[1]{\textcolor[rgb]{0.00,0.44,0.13}{{#1}}}
    \newcommand{\AlertTok}[1]{\textcolor[rgb]{1.00,0.00,0.00}{\textbf{{#1}}}}
    \newcommand{\FunctionTok}[1]{\textcolor[rgb]{0.02,0.16,0.49}{{#1}}}
    \newcommand{\RegionMarkerTok}[1]{{#1}}
    \newcommand{\ErrorTok}[1]{\textcolor[rgb]{1.00,0.00,0.00}{\textbf{{#1}}}}
    \newcommand{\NormalTok}[1]{{#1}}
    
    % Additional commands for more recent versions of Pandoc
    \newcommand{\ConstantTok}[1]{\textcolor[rgb]{0.53,0.00,0.00}{{#1}}}
    \newcommand{\SpecialCharTok}[1]{\textcolor[rgb]{0.25,0.44,0.63}{{#1}}}
    \newcommand{\VerbatimStringTok}[1]{\textcolor[rgb]{0.25,0.44,0.63}{{#1}}}
    \newcommand{\SpecialStringTok}[1]{\textcolor[rgb]{0.73,0.40,0.53}{{#1}}}
    \newcommand{\ImportTok}[1]{{#1}}
    \newcommand{\DocumentationTok}[1]{\textcolor[rgb]{0.73,0.13,0.13}{\textit{{#1}}}}
    \newcommand{\AnnotationTok}[1]{\textcolor[rgb]{0.38,0.63,0.69}{\textbf{\textit{{#1}}}}}
    \newcommand{\CommentVarTok}[1]{\textcolor[rgb]{0.38,0.63,0.69}{\textbf{\textit{{#1}}}}}
    \newcommand{\VariableTok}[1]{\textcolor[rgb]{0.10,0.09,0.49}{{#1}}}
    \newcommand{\ControlFlowTok}[1]{\textcolor[rgb]{0.00,0.44,0.13}{\textbf{{#1}}}}
    \newcommand{\OperatorTok}[1]{\textcolor[rgb]{0.40,0.40,0.40}{{#1}}}
    \newcommand{\BuiltInTok}[1]{{#1}}
    \newcommand{\ExtensionTok}[1]{{#1}}
    \newcommand{\PreprocessorTok}[1]{\textcolor[rgb]{0.74,0.48,0.00}{{#1}}}
    \newcommand{\AttributeTok}[1]{\textcolor[rgb]{0.49,0.56,0.16}{{#1}}}
    \newcommand{\InformationTok}[1]{\textcolor[rgb]{0.38,0.63,0.69}{\textbf{\textit{{#1}}}}}
    \newcommand{\WarningTok}[1]{\textcolor[rgb]{0.38,0.63,0.69}{\textbf{\textit{{#1}}}}}
    
    
    % Define a nice break command that doesn't care if a line doesn't already
    % exist.
    \def\br{\hspace*{\fill} \\* }
    % Math Jax compatability definitions
    \def\gt{>}
    \def\lt{<}
    % Document parameters
    \title{second-conf-call}
    
    
    

    % Pygments definitions
    
\makeatletter
\def\PY@reset{\let\PY@it=\relax \let\PY@bf=\relax%
    \let\PY@ul=\relax \let\PY@tc=\relax%
    \let\PY@bc=\relax \let\PY@ff=\relax}
\def\PY@tok#1{\csname PY@tok@#1\endcsname}
\def\PY@toks#1+{\ifx\relax#1\empty\else%
    \PY@tok{#1}\expandafter\PY@toks\fi}
\def\PY@do#1{\PY@bc{\PY@tc{\PY@ul{%
    \PY@it{\PY@bf{\PY@ff{#1}}}}}}}
\def\PY#1#2{\PY@reset\PY@toks#1+\relax+\PY@do{#2}}

\expandafter\def\csname PY@tok@w\endcsname{\def\PY@tc##1{\textcolor[rgb]{0.73,0.73,0.73}{##1}}}
\expandafter\def\csname PY@tok@c\endcsname{\let\PY@it=\textit\def\PY@tc##1{\textcolor[rgb]{0.25,0.50,0.50}{##1}}}
\expandafter\def\csname PY@tok@cp\endcsname{\def\PY@tc##1{\textcolor[rgb]{0.74,0.48,0.00}{##1}}}
\expandafter\def\csname PY@tok@k\endcsname{\let\PY@bf=\textbf\def\PY@tc##1{\textcolor[rgb]{0.00,0.50,0.00}{##1}}}
\expandafter\def\csname PY@tok@kp\endcsname{\def\PY@tc##1{\textcolor[rgb]{0.00,0.50,0.00}{##1}}}
\expandafter\def\csname PY@tok@kt\endcsname{\def\PY@tc##1{\textcolor[rgb]{0.69,0.00,0.25}{##1}}}
\expandafter\def\csname PY@tok@o\endcsname{\def\PY@tc##1{\textcolor[rgb]{0.40,0.40,0.40}{##1}}}
\expandafter\def\csname PY@tok@ow\endcsname{\let\PY@bf=\textbf\def\PY@tc##1{\textcolor[rgb]{0.67,0.13,1.00}{##1}}}
\expandafter\def\csname PY@tok@nb\endcsname{\def\PY@tc##1{\textcolor[rgb]{0.00,0.50,0.00}{##1}}}
\expandafter\def\csname PY@tok@nf\endcsname{\def\PY@tc##1{\textcolor[rgb]{0.00,0.00,1.00}{##1}}}
\expandafter\def\csname PY@tok@nc\endcsname{\let\PY@bf=\textbf\def\PY@tc##1{\textcolor[rgb]{0.00,0.00,1.00}{##1}}}
\expandafter\def\csname PY@tok@nn\endcsname{\let\PY@bf=\textbf\def\PY@tc##1{\textcolor[rgb]{0.00,0.00,1.00}{##1}}}
\expandafter\def\csname PY@tok@ne\endcsname{\let\PY@bf=\textbf\def\PY@tc##1{\textcolor[rgb]{0.82,0.25,0.23}{##1}}}
\expandafter\def\csname PY@tok@nv\endcsname{\def\PY@tc##1{\textcolor[rgb]{0.10,0.09,0.49}{##1}}}
\expandafter\def\csname PY@tok@no\endcsname{\def\PY@tc##1{\textcolor[rgb]{0.53,0.00,0.00}{##1}}}
\expandafter\def\csname PY@tok@nl\endcsname{\def\PY@tc##1{\textcolor[rgb]{0.63,0.63,0.00}{##1}}}
\expandafter\def\csname PY@tok@ni\endcsname{\let\PY@bf=\textbf\def\PY@tc##1{\textcolor[rgb]{0.60,0.60,0.60}{##1}}}
\expandafter\def\csname PY@tok@na\endcsname{\def\PY@tc##1{\textcolor[rgb]{0.49,0.56,0.16}{##1}}}
\expandafter\def\csname PY@tok@nt\endcsname{\let\PY@bf=\textbf\def\PY@tc##1{\textcolor[rgb]{0.00,0.50,0.00}{##1}}}
\expandafter\def\csname PY@tok@nd\endcsname{\def\PY@tc##1{\textcolor[rgb]{0.67,0.13,1.00}{##1}}}
\expandafter\def\csname PY@tok@s\endcsname{\def\PY@tc##1{\textcolor[rgb]{0.73,0.13,0.13}{##1}}}
\expandafter\def\csname PY@tok@sd\endcsname{\let\PY@it=\textit\def\PY@tc##1{\textcolor[rgb]{0.73,0.13,0.13}{##1}}}
\expandafter\def\csname PY@tok@si\endcsname{\let\PY@bf=\textbf\def\PY@tc##1{\textcolor[rgb]{0.73,0.40,0.53}{##1}}}
\expandafter\def\csname PY@tok@se\endcsname{\let\PY@bf=\textbf\def\PY@tc##1{\textcolor[rgb]{0.73,0.40,0.13}{##1}}}
\expandafter\def\csname PY@tok@sr\endcsname{\def\PY@tc##1{\textcolor[rgb]{0.73,0.40,0.53}{##1}}}
\expandafter\def\csname PY@tok@ss\endcsname{\def\PY@tc##1{\textcolor[rgb]{0.10,0.09,0.49}{##1}}}
\expandafter\def\csname PY@tok@sx\endcsname{\def\PY@tc##1{\textcolor[rgb]{0.00,0.50,0.00}{##1}}}
\expandafter\def\csname PY@tok@m\endcsname{\def\PY@tc##1{\textcolor[rgb]{0.40,0.40,0.40}{##1}}}
\expandafter\def\csname PY@tok@gh\endcsname{\let\PY@bf=\textbf\def\PY@tc##1{\textcolor[rgb]{0.00,0.00,0.50}{##1}}}
\expandafter\def\csname PY@tok@gu\endcsname{\let\PY@bf=\textbf\def\PY@tc##1{\textcolor[rgb]{0.50,0.00,0.50}{##1}}}
\expandafter\def\csname PY@tok@gd\endcsname{\def\PY@tc##1{\textcolor[rgb]{0.63,0.00,0.00}{##1}}}
\expandafter\def\csname PY@tok@gi\endcsname{\def\PY@tc##1{\textcolor[rgb]{0.00,0.63,0.00}{##1}}}
\expandafter\def\csname PY@tok@gr\endcsname{\def\PY@tc##1{\textcolor[rgb]{1.00,0.00,0.00}{##1}}}
\expandafter\def\csname PY@tok@ge\endcsname{\let\PY@it=\textit}
\expandafter\def\csname PY@tok@gs\endcsname{\let\PY@bf=\textbf}
\expandafter\def\csname PY@tok@gp\endcsname{\let\PY@bf=\textbf\def\PY@tc##1{\textcolor[rgb]{0.00,0.00,0.50}{##1}}}
\expandafter\def\csname PY@tok@go\endcsname{\def\PY@tc##1{\textcolor[rgb]{0.53,0.53,0.53}{##1}}}
\expandafter\def\csname PY@tok@gt\endcsname{\def\PY@tc##1{\textcolor[rgb]{0.00,0.27,0.87}{##1}}}
\expandafter\def\csname PY@tok@err\endcsname{\def\PY@bc##1{\setlength{\fboxsep}{0pt}\fcolorbox[rgb]{1.00,0.00,0.00}{1,1,1}{\strut ##1}}}
\expandafter\def\csname PY@tok@kc\endcsname{\let\PY@bf=\textbf\def\PY@tc##1{\textcolor[rgb]{0.00,0.50,0.00}{##1}}}
\expandafter\def\csname PY@tok@kd\endcsname{\let\PY@bf=\textbf\def\PY@tc##1{\textcolor[rgb]{0.00,0.50,0.00}{##1}}}
\expandafter\def\csname PY@tok@kn\endcsname{\let\PY@bf=\textbf\def\PY@tc##1{\textcolor[rgb]{0.00,0.50,0.00}{##1}}}
\expandafter\def\csname PY@tok@kr\endcsname{\let\PY@bf=\textbf\def\PY@tc##1{\textcolor[rgb]{0.00,0.50,0.00}{##1}}}
\expandafter\def\csname PY@tok@bp\endcsname{\def\PY@tc##1{\textcolor[rgb]{0.00,0.50,0.00}{##1}}}
\expandafter\def\csname PY@tok@fm\endcsname{\def\PY@tc##1{\textcolor[rgb]{0.00,0.00,1.00}{##1}}}
\expandafter\def\csname PY@tok@vc\endcsname{\def\PY@tc##1{\textcolor[rgb]{0.10,0.09,0.49}{##1}}}
\expandafter\def\csname PY@tok@vg\endcsname{\def\PY@tc##1{\textcolor[rgb]{0.10,0.09,0.49}{##1}}}
\expandafter\def\csname PY@tok@vi\endcsname{\def\PY@tc##1{\textcolor[rgb]{0.10,0.09,0.49}{##1}}}
\expandafter\def\csname PY@tok@vm\endcsname{\def\PY@tc##1{\textcolor[rgb]{0.10,0.09,0.49}{##1}}}
\expandafter\def\csname PY@tok@sa\endcsname{\def\PY@tc##1{\textcolor[rgb]{0.73,0.13,0.13}{##1}}}
\expandafter\def\csname PY@tok@sb\endcsname{\def\PY@tc##1{\textcolor[rgb]{0.73,0.13,0.13}{##1}}}
\expandafter\def\csname PY@tok@sc\endcsname{\def\PY@tc##1{\textcolor[rgb]{0.73,0.13,0.13}{##1}}}
\expandafter\def\csname PY@tok@dl\endcsname{\def\PY@tc##1{\textcolor[rgb]{0.73,0.13,0.13}{##1}}}
\expandafter\def\csname PY@tok@s2\endcsname{\def\PY@tc##1{\textcolor[rgb]{0.73,0.13,0.13}{##1}}}
\expandafter\def\csname PY@tok@sh\endcsname{\def\PY@tc##1{\textcolor[rgb]{0.73,0.13,0.13}{##1}}}
\expandafter\def\csname PY@tok@s1\endcsname{\def\PY@tc##1{\textcolor[rgb]{0.73,0.13,0.13}{##1}}}
\expandafter\def\csname PY@tok@mb\endcsname{\def\PY@tc##1{\textcolor[rgb]{0.40,0.40,0.40}{##1}}}
\expandafter\def\csname PY@tok@mf\endcsname{\def\PY@tc##1{\textcolor[rgb]{0.40,0.40,0.40}{##1}}}
\expandafter\def\csname PY@tok@mh\endcsname{\def\PY@tc##1{\textcolor[rgb]{0.40,0.40,0.40}{##1}}}
\expandafter\def\csname PY@tok@mi\endcsname{\def\PY@tc##1{\textcolor[rgb]{0.40,0.40,0.40}{##1}}}
\expandafter\def\csname PY@tok@il\endcsname{\def\PY@tc##1{\textcolor[rgb]{0.40,0.40,0.40}{##1}}}
\expandafter\def\csname PY@tok@mo\endcsname{\def\PY@tc##1{\textcolor[rgb]{0.40,0.40,0.40}{##1}}}
\expandafter\def\csname PY@tok@ch\endcsname{\let\PY@it=\textit\def\PY@tc##1{\textcolor[rgb]{0.25,0.50,0.50}{##1}}}
\expandafter\def\csname PY@tok@cm\endcsname{\let\PY@it=\textit\def\PY@tc##1{\textcolor[rgb]{0.25,0.50,0.50}{##1}}}
\expandafter\def\csname PY@tok@cpf\endcsname{\let\PY@it=\textit\def\PY@tc##1{\textcolor[rgb]{0.25,0.50,0.50}{##1}}}
\expandafter\def\csname PY@tok@c1\endcsname{\let\PY@it=\textit\def\PY@tc##1{\textcolor[rgb]{0.25,0.50,0.50}{##1}}}
\expandafter\def\csname PY@tok@cs\endcsname{\let\PY@it=\textit\def\PY@tc##1{\textcolor[rgb]{0.25,0.50,0.50}{##1}}}

\def\PYZbs{\char`\\}
\def\PYZus{\char`\_}
\def\PYZob{\char`\{}
\def\PYZcb{\char`\}}
\def\PYZca{\char`\^}
\def\PYZam{\char`\&}
\def\PYZlt{\char`\<}
\def\PYZgt{\char`\>}
\def\PYZsh{\char`\#}
\def\PYZpc{\char`\%}
\def\PYZdl{\char`\$}
\def\PYZhy{\char`\-}
\def\PYZsq{\char`\'}
\def\PYZdq{\char`\"}
\def\PYZti{\char`\~}
% for compatibility with earlier versions
\def\PYZat{@}
\def\PYZlb{[}
\def\PYZrb{]}
\makeatother


    % Exact colors from NB
    \definecolor{incolor}{rgb}{0.0, 0.0, 0.5}
    \definecolor{outcolor}{rgb}{0.545, 0.0, 0.0}



    
    % Prevent overflowing lines due to hard-to-break entities
    \sloppy 
    % Setup hyperref package
    \hypersetup{
      breaklinks=true,  % so long urls are correctly broken across lines
      colorlinks=true,
      urlcolor=urlcolor,
      linkcolor=linkcolor,
      citecolor=citecolor,
      }
    % Slightly bigger margins than the latex defaults
    
    \geometry{verbose,tmargin=1in,bmargin=1in,lmargin=1in,rmargin=1in}
    
    

    \begin{document}
    
    
    \maketitle
    
    

    
    \section{Second conf call fast
report}\label{second-conf-call-fast-report}

    \begin{Verbatim}[commandchars=\\\{\}]
{\color{incolor}In [{\color{incolor}1}]:} \PY{k+kn}{import} \PY{n+nn}{datetime}
        \PY{n+nb}{print}\PY{p}{(}\PY{n+nb}{str}\PY{p}{(}\PY{n}{datetime}\PY{o}{.}\PY{n}{datetime}\PY{o}{.}\PY{n}{today}\PY{p}{(}\PY{p}{)}\PY{p}{)}\PY{p}{)}
\end{Verbatim}


    \begin{Verbatim}[commandchars=\\\{\}]
2018-04-27 09:52:28.261868

    \end{Verbatim}

    \emph{For this second report I am experimenting a new way to fastly
deploy updates about my project: \textbf{Jupyter notebook}. In this
stage of the project, I'd like to share more code snippets than I would
in the final thesis, hence markdown support and live execution of code
snippets turned out to be very useful for such purpose. Actually the two
main reason why I decide to move from raw LaTeX to here are: (1) Jupyter
does it faster, (2) it convert everything to LaTeX (I won't be required
to make any effort to include everything from here to my LaTeX thesis
document). Whether, for any reason, this method will turn out to be
ineffective or time-expensive I will drop it out.}

\subsection{Change log summary}\label{change-log-summary}

\subsubsection{Training set generator
function}\label{training-set-generator-function}

Now all samples components domains are centered at the same value \(c\).
Such domains are determined as follows.

Is defined \(K = \{k_1,...k_m\}\) where \(k_i\) is the domain radius of
variable \(x_i\). Let \(x=(x_1,...,x_m) \in X\) be a sample of the
training set, then the domain of \(x_i\) is \(D(x_i) = [c-k_i, c+k_i]\).

\begin{Shaded}
\begin{Highlighting}[]
\KeywordTok{def}\NormalTok{ sample_from_function(n_samples, n_features, func, domain_radius}\OperatorTok{=}\FloatTok{0.5}\NormalTok{, domain_center}\OperatorTok{=}\FloatTok{0.5}\NormalTok{,error_mean}\OperatorTok{=}\DecValTok{0}\NormalTok{, error_std_dev}\OperatorTok{=}\DecValTok{1}\NormalTok{):}
\NormalTok{    X }\OperatorTok{=}\NormalTok{ []}
\NormalTok{    y }\OperatorTok{=}\NormalTok{ np.zeros(n_samples)}
\NormalTok{    w }\OperatorTok{=}\NormalTok{ np.ones(n_features)}
\NormalTok{    K }\OperatorTok{=}\NormalTok{ np.random.uniform(domain_center }\OperatorTok{-}\NormalTok{ domain_radius, domain_center }\OperatorTok{+}\NormalTok{ domain_radius, n_features)}

    \ControlFlowTok{for}\NormalTok{ i }\KeywordTok{in} \BuiltInTok{range}\NormalTok{(n_samples):}
\NormalTok{        x }\OperatorTok{=}\NormalTok{ np.zeros(n_features)}
        \ControlFlowTok{for}\NormalTok{ j }\KeywordTok{in} \BuiltInTok{range}\NormalTok{(n_features):}
\NormalTok{            x[j] }\OperatorTok{=}\NormalTok{ np.random.uniform(}\OperatorTok{-}\NormalTok{K[j] }\OperatorTok{+}\NormalTok{ domain_radius, K[j] }\OperatorTok{+}\NormalTok{ domain_radius)}
\NormalTok{        X.append(x)}
\NormalTok{        y[i] }\OperatorTok{=}\NormalTok{ func(x, w) }\OperatorTok{+}\NormalTok{ np.random.normal(error_mean, error_std_dev)}

\ControlFlowTok{return}\NormalTok{ np.array(X), y}
\end{Highlighting}
\end{Shaded}

    \textbf{Example}. Detailed execution of \texttt{sample\_from\_function}
with the following parameters: - \texttt{n\_samples\ =\ 8} : amount of
samples in the training set; - \texttt{n\_features\ =\ 4} : number of
features, e.g. size of each array sample; - \texttt{func} : classic
linear function \(f(\vec{x},\vec{w})=\sum_{j=1}^{m} x_j w_j\) where
\(m\) is the size of \(x\) (and \(w\)), so, for instance,
\(f((2,3),(4,1))=2*4+3*1=11\); - \texttt{domain\_radius\ =\ 5}; -
\texttt{domain\_center\ =\ 0}; - \texttt{error\_mean\ =\ 0}; -
\texttt{error\_std\_dev\ =\ 1}: the error (noise) follows

    \begin{Verbatim}[commandchars=\\\{\}]
{\color{incolor}In [{\color{incolor}20}]:} \PY{k+kn}{import} \PY{n+nn}{numpy} \PY{k}{as} \PY{n+nn}{np}
         \PY{k+kn}{import} \PY{n+nn}{pprint}
         \PY{n}{n\PYZus{}samples} \PY{o}{=} \PY{l+m+mi}{8}
         \PY{n}{n\PYZus{}features} \PY{o}{=} \PY{l+m+mi}{4}
         \PY{n}{func} \PY{o}{=} \PY{k}{lambda} \PY{n}{\PYZus{}X}\PY{p}{,}\PY{n}{\PYZus{}w} \PY{p}{:} \PY{n}{\PYZus{}X}\PY{o}{.}\PY{n}{dot}\PY{p}{(}\PY{n}{\PYZus{}w}\PY{p}{)}
         \PY{n}{domain\PYZus{}center} \PY{o}{=} \PY{l+m+mi}{0}
         \PY{n}{domain\PYZus{}radius} \PY{o}{=} \PY{l+m+mi}{5}
         \PY{n}{error\PYZus{}mean} \PY{o}{=} \PY{l+m+mi}{0}
         \PY{n}{error\PYZus{}std\PYZus{}dev} \PY{o}{=} \PY{l+m+mi}{1}
         \PY{n}{X} \PY{o}{=} \PY{p}{[}\PY{p}{]}
         \PY{n}{y} \PY{o}{=} \PY{n}{np}\PY{o}{.}\PY{n}{zeros}\PY{p}{(}\PY{n}{n\PYZus{}samples}\PY{p}{)}
         \PY{n}{w} \PY{o}{=} \PY{n}{np}\PY{o}{.}\PY{n}{ones}\PY{p}{(}\PY{n}{n\PYZus{}features}\PY{p}{)}
         \PY{n}{K} \PY{o}{=} \PY{n}{np}\PY{o}{.}\PY{n}{random}\PY{o}{.}\PY{n}{uniform}\PY{p}{(}\PY{n}{domain\PYZus{}center} \PY{o}{\PYZhy{}} \PY{n}{domain\PYZus{}radius}\PY{p}{,} \PY{n}{domain\PYZus{}center} \PY{o}{+} \PY{n}{domain\PYZus{}radius}\PY{p}{,} \PY{n}{n\PYZus{}features}\PY{p}{)}
\end{Verbatim}


    \(K = \{k_1,...k_m\}\) where \(k_i\) is the domain radius of variable
\(x_i\), so the domain of \(x_i\) will be \(D(x_i) = [c-k_i, c+k_i]\)
where \(c\) is the center of the domain (\texttt{domain\_center}); \(c\)
is the same for all \(x_i\).

    \begin{Verbatim}[commandchars=\\\{\}]
{\color{incolor}In [{\color{incolor}19}]:} \PY{n+nb}{print}\PY{p}{(}\PY{n}{K}\PY{p}{)}
\end{Verbatim}


    \begin{Verbatim}[commandchars=\\\{\}]
[1.32375564 0.71056962 2.54217545 1.6100447 ]

    \end{Verbatim}

    \begin{Verbatim}[commandchars=\\\{\}]
{\color{incolor}In [{\color{incolor}18}]:} \PY{k}{for} \PY{n}{i} \PY{o+ow}{in} \PY{n+nb}{range}\PY{p}{(}\PY{n}{n\PYZus{}samples}\PY{p}{)}\PY{p}{:}
             \PY{n}{x} \PY{o}{=} \PY{n}{np}\PY{o}{.}\PY{n}{zeros}\PY{p}{(}\PY{n}{n\PYZus{}features}\PY{p}{)}
             \PY{k}{for} \PY{n}{j} \PY{o+ow}{in} \PY{n+nb}{range}\PY{p}{(}\PY{n}{n\PYZus{}features}\PY{p}{)}\PY{p}{:}
                 \PY{n}{x}\PY{p}{[}\PY{n}{j}\PY{p}{]} \PY{o}{=} \PY{n}{np}\PY{o}{.}\PY{n}{random}\PY{o}{.}\PY{n}{uniform}\PY{p}{(}\PY{o}{\PYZhy{}}\PY{n}{K}\PY{p}{[}\PY{n}{j}\PY{p}{]} \PY{o}{+} \PY{n}{domain\PYZus{}radius}\PY{p}{,} \PY{n}{K}\PY{p}{[}\PY{n}{j}\PY{p}{]} \PY{o}{+} \PY{n}{domain\PYZus{}radius}\PY{p}{)}
             \PY{n}{X}\PY{o}{.}\PY{n}{append}\PY{p}{(}\PY{n}{x}\PY{p}{)}
             \PY{n}{y}\PY{p}{[}\PY{n}{i}\PY{p}{]} \PY{o}{=} \PY{n}{func}\PY{p}{(}\PY{n}{x}\PY{p}{,} \PY{n}{w}\PY{p}{)} \PY{o}{+} \PY{n}{np}\PY{o}{.}\PY{n}{random}\PY{o}{.}\PY{n}{normal}\PY{p}{(}\PY{n}{error\PYZus{}mean}\PY{p}{,} \PY{n}{error\PYZus{}std\PYZus{}dev}\PY{p}{)}
         \PY{n+nb}{print}\PY{p}{(}\PY{n}{pprint}\PY{o}{.}\PY{n}{PrettyPrinter}\PY{p}{(}\PY{n}{indent}\PY{o}{=}\PY{l+m+mi}{4}\PY{p}{)}\PY{o}{.}\PY{n}{pformat}\PY{p}{(}\PY{n}{X}\PY{p}{)}\PY{p}{)}
         \PY{n+nb}{print}\PY{p}{(}\PY{n}{pprint}\PY{o}{.}\PY{n}{PrettyPrinter}\PY{p}{(}\PY{n}{indent}\PY{o}{=}\PY{l+m+mi}{4}\PY{p}{)}\PY{o}{.}\PY{n}{pformat}\PY{p}{(}\PY{n}{y}\PY{p}{)}\PY{p}{)}
\end{Verbatim}


    \begin{Verbatim}[commandchars=\\\{\}]
[   array([6.02188807, 5.67014992, 4.96419302, 5.14972292]),
    array([5.9642668 , 4.75911729, 3.1911183 , 6.48086813]),
    array([3.77524004, 5.69812223, 3.93359248, 4.15090944]),
    array([5.89155598, 4.33238006, 4.67047585, 6.58797714]),
    array([5.78911644, 4.47464001, 6.09150392, 4.30422703]),
    array([6.09066085, 5.44278406, 7.06243236, 5.28890016]),
    array([5.48128054, 4.92461966, 6.29826249, 4.08255381]),
    array([4.33756388, 5.65386185, 6.73384508, 6.09020554])]
array([23.09383521, 20.77253317, 17.89046891, 21.22414156, 20.48504975,
       24.51765509, 19.07431732, 23.32166701])

    \end{Verbatim}

    \subsubsection{New metric Real Mean Squared Error
(RMSE)}\label{new-metric-real-mean-squared-error-rmse}

The training set is in the form of a pair (X,y) where \(y_i\) is the
value the target function is supposed to yield for the input \(x_i\),
actually, whether a noise exists in the training set, then \(y_i\)
differs from the real value \(\tilde{y}_i\) by a gap that, in our case,
is normally distributed (with mean = \texttt{error\_mean} and standard
deviation = \texttt{error\_std\_dev}). Whereas the training set is
generated with fully control over all parameters, then we know either
the perturbed value \(y_i\) either the real value
\(\tilde{y}_i=\mathbb{1} x_i=\sum_{j=1}^{m}x_{ij}\), so we have an
additional information to take into account in order to study the
behaviour of different models.

While the mean squared error
\[MSE = \frac{1}{N} \sum_{i=1}^{n} (y_i - \hat{y}_i) (\hat{y}_i)'\]
regards \(y_i\), the real MSE concerns \(\tilde{y}_i\), hence it's
defined as
\[RMSE = \frac{1}{N} \sum_{i=1}^{n} (\tilde{y}_i - \hat{y}_i) (\hat{y}_i)'.\]

By comparing these two metrics one can understand whether and how much
the prediction model suffers from noise fitting, e.g. when the model
adapts itself much more on the noise rather than on the provided target
function value.

Obviously, noiseless training sets lead the MSE to be equal to the RMSE.

    \subsubsection{Computing the error over the whole training
set}\label{computing-the-error-over-the-whole-training-set}

After each iteration each node computes a set of metrics taking into
account its local model and knowledge, so each node keeps track of the
history of its weight vector, MAE (mean absolute error), RMAE (to be
implemented yet), MSE, RMSE.

Previously I computed the error as the mean of the MSE of each node. A
way that, at first glance, could seem to be reasonable, but, actually,
it is not: such MSE is not computed with one weight vector over the
entire training set, indeed if someone asked for the weight vector which
had produced such result, then we haven't a value to provide they with.
That's why the correct way to compute the global metrics is to retrieve
from each node \(k\) its local model \(\vec{w}_{k}\), compute
\(\vec{w} = \frac{1}{K}\sum_{k=1}^{K}\vec{w}_k\) and then computes
metrics taking into account \(\vec{w}\) along with the whole training
set.

    \subsubsection{Test description}\label{test-description}

Since there are several parameters to set up in order to run a test, I
have excluded to pass them from the command line (by running the program
in a way like \texttt{\$\ python\ main.py\ {[}parameters{]}}), instead
they're are set directly inside the script \texttt{main.py}. Leave aside
for a while the system and training task setup, there are many other
settings which ensure control over the test execution and outputs.
Without deepening their implementation, when a new test is run, the
simulator creates a new folder \texttt{/\$TEST\_NAME} in
\texttt{/test\_log} that contains: - a \texttt{/plot} folder with all
plots images; - all global logs of the simulation for each topology; -
the descriptor file that reports the detailed parameters values used for
the test; - a serialization of the setup that can be used to run again
the same simulation.

\textbf{Example}.

\begin{verbatim}
./test_log
└── /$TEST_NAME
    ├── /plot
    │   ├── iter_time.png
    │   ├── mse_iter.png
    │   ├── mse_time.png
    │   ├── real-mse_iter.png
    │   └── real-mse_time.png
    ├── clique_global_mean_squared_error_log
    ├── clique_global_real_mean_squared_error_log
    ├── clique_iterations_time_log
    ├── cycle_global_mean_squared_error_log
    ├── cycle_global_real_mean_squared_error_log
    ├── cycle_iterations_time_log
    ├── diagonal_global_mean_squared_error_log
    ├── diagonal_global_real_mean_squared_error_log
    ├── diagonal_iterations_time_log
    ├── diam-expander_global_mean_squared_error_log
    ├── diam-expander_global_real_mean_squared_error_log
    ├── diam-expander_iterations_time_log
    ├── root-expander_global_mean_squared_error_log
    ├── root-expander_global_real_mean_squared_error_log
    ├── root-expander_iterations_time_log
    ├── .descriptor.txt
    └── .setup.pkl
\end{verbatim}

\paragraph{Test descriptor}\label{test-descriptor}

Below is how the descriptor appears, it doesn't matter if some or most
of them could be not immediately understandable, simply take a fast look
at this code snippet to realize how many parameters the system let us
customize.

\begin{verbatim}
>>> Test Descriptor File
Title: test
Date: 2018-04-26 11:28:39.116108
Summary: -

### BEGIN SETUP ###
n = 10
seed = 1524734919
graphs = {
    'clique': np.array([
        [1., 1., 1., 1., 1., 1., 1., 1., 1., 1.],
        [1., 1., 1., 1., 1., 1., 1., 1., 1., 1.],
        [1., 1., 1., 1., 1., 1., 1., 1., 1., 1.],
        [1., 1., 1., 1., 1., 1., 1., 1., 1., 1.],
        [1., 1., 1., 1., 1., 1., 1., 1., 1., 1.],
        [1., 1., 1., 1., 1., 1., 1., 1., 1., 1.],
        [1., 1., 1., 1., 1., 1., 1., 1., 1., 1.],
        [1., 1., 1., 1., 1., 1., 1., 1., 1., 1.],
        [1., 1., 1., 1., 1., 1., 1., 1., 1., 1.],
        [1., 1., 1., 1., 1., 1., 1., 1., 1., 1.]]),
    'cycle': np.array([
        [1., 1., 0., 0., 0., 0., 0., 0., 0., 0.],
        [0., 1., 1., 0., 0., 0., 0., 0., 0., 0.],
        [0., 0., 1., 1., 0., 0., 0., 0., 0., 0.],
        [0., 0., 0., 1., 1., 0., 0., 0., 0., 0.],
        [0., 0., 0., 0., 1., 1., 0., 0., 0., 0.],
        [0., 0., 0., 0., 0., 1., 1., 0., 0., 0.],
        [0., 0., 0., 0., 0., 0., 1., 1., 0., 0.],
        [0., 0., 0., 0., 0., 0., 0., 1., 1., 0.],
        [0., 0., 0., 0., 0., 0., 0., 0., 1., 1.],
        [1., 0., 0., 0., 0., 0., 0., 0., 0., 1.]]),
    'diagonal': np.array([
        [1., 0., 0., 0., 0., 0., 0., 0., 0., 0.],
        [0., 1., 0., 0., 0., 0., 0., 0., 0., 0.],
        [0., 0., 1., 0., 0., 0., 0., 0., 0., 0.],
        [0., 0., 0., 1., 0., 0., 0., 0., 0., 0.],
        [0., 0., 0., 0., 1., 0., 0., 0., 0., 0.],
        [0., 0., 0., 0., 0., 1., 0., 0., 0., 0.],
        [0., 0., 0., 0., 0., 0., 1., 0., 0., 0.],
        [0., 0., 0., 0., 0., 0., 0., 1., 0., 0.],
        [0., 0., 0., 0., 0., 0., 0., 0., 1., 0.],
        [0., 0., 0., 0., 0., 0., 0., 0., 0., 1.]]),
    'diam-expander': np.array([
        [1., 1., 0., 0., 0., 1., 0., 0., 0., 1.],
        [1., 1., 1., 0., 0., 0., 1., 0., 0., 0.],
        [0., 1., 1., 1., 0., 0., 0., 1., 0., 0.],
        [0., 0., 1., 1., 1., 0., 0., 0., 1., 0.],
        [0., 0., 0., 1., 1., 1., 0., 0., 0., 1.],
        [1., 0., 0., 0., 1., 1., 1., 0., 0., 0.],
        [0., 1., 0., 0., 0., 1., 1., 1., 0., 0.],
        [0., 0., 1., 0., 0., 0., 1., 1., 1., 0.],
        [0., 0., 0., 1., 0., 0., 0., 1., 1., 1.],
        [1., 0., 0., 0., 1., 0., 0., 0., 1., 1.]]),
    'root-expander': np.array([
        [1., 1., 0., 1., 0., 0., 0., 0., 0., 0.],
        [0., 1., 1., 0., 1., 0., 0., 0., 0., 0.],
        [0., 0., 1., 1., 0., 1., 0., 0., 0., 0.],
        [0., 0., 0., 1., 1., 0., 1., 0., 0., 0.],
        [0., 0., 0., 0., 1., 1., 0., 1., 0., 0.],
        [0., 0., 0., 0., 0., 1., 1., 0., 1., 0.],
        [0., 0., 0., 0., 0., 0., 1., 1., 0., 1.],
        [1., 0., 0., 0., 0., 0., 0., 1., 1., 0.],
        [0., 1., 0., 0., 0., 0., 0., 0., 1., 1.],
        [1., 0., 1., 0., 0., 0., 0., 0., 0., 1.]]),
    'star': np.array([
        [1., 1., 1., 1., 1., 1., 1., 1., 1., 1.],
        [1., 1., 0., 0., 0., 0., 0., 0., 0., 0.],
        [1., 0., 1., 0., 0., 0., 0., 0., 0., 0.],
        [1., 0., 0., 1., 0., 0., 0., 0., 0., 0.],
        [1., 0., 0., 0., 1., 0., 0., 0., 0., 0.],
        [1., 0., 0., 0., 0., 1., 0., 0., 0., 0.],
        [1., 0., 0., 0., 0., 0., 1., 0., 0., 0.],
        [1., 0., 0., 0., 0., 0., 0., 1., 0., 0.],
        [1., 0., 0., 0., 0., 0., 0., 0., 1., 0.],
        [1., 0., 0., 0., 0., 0., 0., 0., 0., 1.]]
    )}

# TRAINING SET SETUP
n_samples = 100000
n_features = 100
yhat = <class 'src.mltoolbox.LinearYHatFunction'>
domain_radius = 5
domain_center = 0
error_mean = 0
error_std_dev = 0

# CLUSTER SETUP
sample_function = <function LinearYHatFunction.f at 0x7f13fa4d00d0>
max_iter = None
max_time = 100000
method = stochastic
batch_size = 20
activation_func = None
loss = <class 'src.mltoolbox.SquaredLossFunction'>
penalty = l2
epsilon = 0.01
alpha = 1e-06
learning_rate = constant
metrics = all
alt_metrics = False
shuffle = True
verbose = False
\end{verbatim}


    % Add a bibliography block to the postdoc
    
    
    
    \end{document}
